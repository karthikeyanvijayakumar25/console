\hypertarget{group__flashd__module}{}\section{Flash Memory Interface}
\label{group__flashd__module}\index{Flash Memory Interface@{Flash Memory Interface}}
The flash driver manages the programming, erasing, locking and unlocking sequences with dedicated commands.

To implement flash programming operation, the user has to follow these few steps \+: 
\begin{DoxyItemize}
\item Configure flash wait states to initializes the flash.  
\item Checks whether a region to be programmed is locked.  
\item Unlocks the user region to be programmed if the region have locked before. 
\item Erases the user page before program (optional). 
\item Writes the user page from the page buffer. 
\item Locks the region of programmed area if any. 
\end{DoxyItemize}

Writing 8-\/bit and 16-\/bit data is not allowed and may lead to unpredictable data corruption. A check of this validity and padding for 32-\/bit alignment should be done in write algorithm. Lock/unlock range associated with the user address range is automatically translated.

This security bit can be enabled through the command \char`\"{}\+Set General Purpose
\+N\+V\+M Bit 0\char`\"{}.

A 128-\/bit factory programmed unique ID could be read to serve several purposes.

The driver accesses the flash memory by calling the lowlevel module provided in \mbox{\hyperlink{group__efc__module}{Working with E\+E\+FC}}. For more accurate information, please look at the E\+E\+FC section of the Datasheet.

Related files \+:~\newline
flashd.\+c~\newline
 flashd.\+h.~\newline
 efc.\+c~\newline
 efc.\+h.~\newline
 